\section{插值法}

\subsection{拉格朗日插值法}

n=1时称为线性插值,n=2时称为抛物插值

基函数
\[
    l_i(x) = \frac{(x-x_0)(x-x_1)\cdots(x-x_{i-1})(x-x_{i+1})\cdots(x-x_n)}{(x_i-x_0)(x_i-x_1)\cdots(x_i-x_{i-1})(x_i-x_{i+1})\cdots(x_i-x_n)}
\]

插值函数
\[
    L_n(x) = \sum_{i=0}^{n}y_il_i(x)
\]

误差
\[
    R_n(x) = \frac{f^{(n+1)}(\xi)}{(n+1)!}\omega_{n+1}(x), \quad \omega_{n+1}(x) = (x-x_0)(x-x_1)\cdots(x-x_n)
\]

计算时用这个误差
\[
|R_n(x)| \leq \frac{M_{n+1}}{(n+1)!}|\omega_{n+1}(x)|, \quad M_{n+1} = \max_{x_0 \leq x \leq x_n}|f^{(n+1)}(x)|
\]

\subsection{牛顿插值法}

n阶差商,有个注意的点是$f^{(n)}$求导为0的时候,差商就是0
\[
    f[x_0,x_1,\cdots,x_n] = \frac{f^{(n)}(\xi)}{n!}
\]

解题方法:列出均差表,计算均差,然后写出插值多项式$$P_n(x) = f[x_0] + f[x_0,x_1](x-x_0) + f[x_0,x_1,x_2](x-x_0)(x-x_1) + \cdots$$

\begin{table}[H]
    \centering
    \begin{tabular}{|c|c|c|c|c|c|}
        \hline
        $x$ & $f(x)$ & $f[x_0,x_1]$ & $f[x_0,x_1,x_2]$ & $f[x_0,x_1,x_2,x_3]$ & $f[x_0,x_1,x_2,x_3,x_4] $\\
        \hline
        $1$ & $1$ &  & & & \\
        $2$ & $4$ & $\frac{4-1}{2-1}=3=C_1$ & & & \\
        $3$ & $7$ & $\frac{7-4}{3-2}=3$ & $\frac{3-3}{3-1}=0=C_2$ & & \\
        $4$ & $8$ & $\frac{8-7}{4-3}=1$ & $\frac{1-3}{4-2}=-1$ & $\frac{-1-0}{4-1}=-1/3=C_3$ & \\
        $5$ & $6$ & $\frac{6-8}{5-4}=2$ & $\frac{2-1}{5-3}=3/2$ & $\frac{3/2+1}{5-2}=-1/6$ & $\frac{-1/6+1/3}{5-1}=1/24=C_4$ \\
        \hline
    \end{tabular}
    \caption{差商表}
\end{table}

误差(和拉格朗日一样)

\subsection{埃尔米特插值法}

与牛顿插值法类似,区别在于一阶导数需要重复2次,二阶导数需要重复3次

计算时遇到0/0的情况就替换成对应的$\frac{f^{(n)}(x)}{n!}$

插值多项式与牛顿插值法类似,替换成$(x-z_0)(x-z_1)\cdots$
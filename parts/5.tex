\section{第5章解线性方程组的直接方法}

本章要求掌握\textbf{列主元消去法}、\textbf{直接三角分解}、\textbf{改进的平方根法}三种解线性方程组的方法,以及求矩阵$A$的4种范数(行范数、列范数、2-范数、F-范数),2种条件数($condA_\infty$、$condA_2$)

\subsection{列主元消去法}

\subsubsection{方法}

题目:解线性方程组并求出$detA$的值

通过换行使所在列的第一个元素绝对值最大(前提能简便计算),\textbf{换一次要在求$detA$时乘以$(-1)$}

初等行变换构造上三角矩阵,从最下面一行求解得到线性方程组的解

$detA=$上三角矩阵对角线元素相乘,再乘以换行时需要的$(-1)$

\subsubsection{例题}

\[
\begin{aligned}
(A|b) = &
\left[
\begin{array}{ccc|c}
12 & -3 & 3 & 15 \\
-18 & 3 & -1 & -15 \\
1 & 1 & 1 & 6
\end{array}
\right] 
\xrightarrow{r_1 \leftrightarrow r_2} 
\left[
\begin{array}{ccc|c}
-18 & 3 & -1 & -15 \\
12 & -3 & 3 & 15 \\
1 & 1 & 1 & 6
\end{array}
\right]
\underset{r_3 + \left( -\frac{1}{18} \right) r_1}{\overset{r_2 + \left( -\frac{2}{3} \right) r_1}{\xrightarrow{\hspace{2cm}}}} \\
&\left[
\begin{array}{ccc|c}
-18 & 3 & -1 & -15 \\
0 & -1 & 7/3 & 5 \\
0 & 7/6 & 17/18 & 31/6
\end{array}
\right]
\xrightarrow{r_3 + \left( -\frac{7}{6} \right) r_2} 
\left[
\begin{array}{ccc|c}
-18 & 3 & -1 & -15 \\
0 & -1 & 7/3 & 5 \\
0 & 0 & 11/3 & 11
\end{array}
\right]
\end{aligned}
\]

由第三行$\frac{11}{3}x_3=11$,得$x_3=3$;

由第二行$-x_2+\frac{7}{3}x_3=5$,得$x_2=2$;

由第一行$-18x_1+3x_2-x_3=-15$,得$x_1=1$.

一次行交换使行列式符号乘以$(-1)$,即$det(A)=(-18)\times(-1)\times\frac{11}{3}\times(-1)=-66$


\subsection{直接三角分解}

\subsubsection{方法}

计算得所有顺序主子式不等于0才可以使用此方法

\[
\begin{aligned}
    A=
    \left[
    \begin{array}{ccc}
        1 & 0 & 0 \\
        l_{21} & 1 & 0 \\
        l_{31} & l_{32} & 1
    \end{array}
    \right]
    \left[
    \begin{array}{ccc}
        u_{11} & u_{12} & u_{13} \\
        0 & u_{22} & u_{23} \\
        0 & 0 & u_{33}
    \end{array}
    \right]
    =LU
    \quad
\end{aligned}
\]

代入具体数值可以求出$L,U$,根据$Ly=B$求出$y$,再根据$Ux=y$可以求出$x$,即线性方程组的解

\subsubsection{例题}

\[
\begin{aligned}
    A=
    \left[
    \begin{array}{ccc}
        1/4 & 1/5 & 1/6 \\
        1/3 & 1/4 & 1/5 \\
        1/2 & 1 & 2
    \end{array}
    \right]
    =
    \left[
    \begin{array}{ccc}
        1 & 0 & 0 \\
        l_{21} & 1 & 0 \\
        l_{31} & l_{32} & 1
    \end{array}
    \right]
    \left[
    \begin{array}{ccc}
        u_{11} & u_{12} & u_{13} \\
        0 & u_{22} & u_{23} \\
        0 & 0 & u_{33}
    \end{array}
    \right]
    \quad
    B=
    \left[
    \begin{array}{c}
        9  \\
        8  \\
        8 
    \end{array}
    \right]
\end{aligned}
\]

\[
u_{11}=\frac{1}{4}, \quad u_{12}=\frac{1}{5}, \quad u_{13}=\frac{1}{6}
\]
\[
l_{21}u_{11}=\frac{1}{3}\xrightarrow{}l_{21}=\frac{4}{3}, \quad l_{21}u_{12}+u_{22}=\frac{1}{4}\xrightarrow{}u_{22}=-\frac{1}{60}, \quad l_{21}u_{13}+u_{23}=\frac{1}{5}\xrightarrow{}u_{23}=-\frac{1}{45}
\]
\[
l_{31}u_{11}=\frac{1}{2}\xrightarrow{}l_{31}=2, \quad l_{31}u_{12}+l_{32}u_{22}=1\xrightarrow{}l_{32}=-36, \quad l_{31}u_{13}+l_{32}u_{23}+u_{33}=2\xrightarrow{}u_{33}=\frac{13}{15}
\]

\[
\begin{aligned}
    L=
    \left[
    \begin{array}{ccc}
        1 & 0 & 0 \\
        \frac{4}{3} & 1 & 0 \\
        2 & -36 & 1
    \end{array}
    \right]
    \quad
    U=
    \left[
    \begin{array}{ccc}
        \frac{1}{4} & \frac{1}{5} & \frac{1}{6} \\
        0 & -\frac{1}{60} & -\frac{1}{45} \\
        0 & 0 & \frac{13}{15}
    \end{array}
    \right]
\end{aligned}
\]
$Ly=B$
\[
\begin{aligned}
    \left[
    \begin{array}{ccc}
        1 & 0 & 0 \\
        \frac{4}{3} & 1 & 0 \\
        2 & -36 & 1
    \end{array}
    \right]
    \left[
    \begin{array}{c}
        y_1  \\
        y_2  \\
        y_3 
    \end{array}
    \right]
    =
    \left[
    \begin{array}{c}
        9  \\
        8  \\
        8 
    \end{array}
    \right]
    \Rightarrow{}
    \left[
    \begin{array}{c}
        y_1  \\
        y_2  \\
        y_3 
    \end{array}
    \right]
    =
    \left[
    \begin{array}{c}
        9  \\
        -4  \\
        -154 
    \end{array}
    \right]
\end{aligned}
\]
$Ux=y$
\[
\begin{aligned}
    \left[
    \begin{array}{ccc}
        \frac{1}{4} & \frac{1}{5} & \frac{1}{6} \\
        0 & -\frac{1}{60} & -\frac{1}{45} \\
        0 & 0 & \frac{13}{15}
    \end{array}
    \right]
    \left[
    \begin{array}{c}
        x_1  \\
        x_2  \\
        x_3 
    \end{array}
    \right]
    =
    \left[
    \begin{array}{c}
        9  \\
        -4  \\
        -154 
    \end{array}
    \right]
    \Rightarrow{}
    \left[
    \begin{array}{c}
        x_1  \\
        x_2  \\
        x_3 
    \end{array}
    \right]
    =
    \left[
    \begin{array}{c}
        -\frac{2952}{13}  \\
        \frac{6200}{13}  \\
        -\frac{2310}{13} 
    \end{array}
    \right]
\end{aligned}
\]

\subsection{改进的平方根法}

\[
\begin{aligned}
    A
    &= 
    \left[
    \begin{array}{ccc}
        1 & 0 & 0 \\
        l_{21} & 1 & 0 \\
        l_{31} & l_{32} & 1
    \end{array}
    \right]
    \left[
    \begin{array}{ccc}
        d_{1} & 0 & 0 \\
        0 & d_{2} &0 \\
        0 & 0 & d_{3}
    \end{array}
    \right]
    \left[
    \begin{array}{ccc}
        1 & l_{21} & l_{31} \\
        0 & 1 & l_{32} \\
        0 & 0 & 1
    \end{array}
    \right] 
    =LDL^T \\
    &=
    \left[
    \begin{array}{ccc}
        1 & 0 & 0 \\
        l_{21} & 1 & 0 \\
        l_{31} & l_{32} & 1
    \end{array}
    \right]
    \left[
    \begin{array}{ccc}
        d_{1} & d_{1}l_{21} & d_{1}l_{31} \\
        0 & d_{2} & d_{2}l_{32} \\
        0 & 0 & d_{3}
    \end{array}
    \right]
\end{aligned}
\]

代入具体数值可以求出$L,D,L^T$,可以把$DL^T$看成前面的$U$,步骤和直接三角分解的一样

$Ly=B$得到$y$,代入$DL^Tx=y$可以求出$x$,即线性方程组的解

\subsection{四种范数}\label{fanshu}

\textbf{行范数} \quad 每一行元素绝对值相加的最大值
    \[
    ||A||_{\infty}=\max_{1 \leq i \leq n} \sum_{j=1}^{n} |a_{ij}|
    \]

    
\textbf{列范数} \quad 每一列元素绝对值相加的最大值
    \[
    ||A||_{1}=\max_{1 \leq j \leq n} \sum_{i=1}^{n} |a_{ij}|
    \]

\textbf{2-范数} \quad $\lambda_{\max}(A^T A)$是指$A^T A$特征值最大值,也就是特征值最大值开根号
    \[
    \|A\|_2 = \sqrt{\lambda_{\max}(A^T A)}
    \]

\textbf{F-范数} \quad 各元素平方和开根号
    \[
    \|A\|_F = \sqrt{\sum_{i=1,j=1}^{n} a_{ij}^2}
    \]


\subsection{两种条件数}

\textbf{$condA_\infty$}

\[
condA_\infty = \|A^{-1}\|_{\infty} \|A\|_{\infty}
\]

\textbf{$condA_2$}

\[
condA_2 = \|A\|_{2} \|A^{-1}\|_{2} = \sqrt{\frac{\lambda_{\max}(A^T A)}{\lambda_{\min}(A A^T)}}
\]
特别的,当$A$为对称矩阵时
\[
condA_2 = \frac{|\lambda_{1}|}{|\lambda_{n}|}
\]
$\lambda_1,\lambda_n$分别为$A$的绝对值最大和绝对值最小的特征值

\subsection{补充知识}

\textbf{求解特征值示例}
\[
\det(A - \lambda I) = 
\begin{vmatrix}
100-\lambda & 99 \\[6pt]
99 & 98-\lambda 
\end{vmatrix}
= 0.
\]

\[
(100-\lambda)(98-\lambda) - 99^2 = 0.
\]

\[
\lambda^2 - 198\lambda - 1 = 0.
\]

\[
\lambda = \frac{198 \pm \sqrt{198^2+4}}{2}.
\]
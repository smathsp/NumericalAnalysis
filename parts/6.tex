\section{第6章解线性方程的迭代法}

本章要求掌握\textbf{雅可比迭代法}、\textbf{高斯-塞德尔迭代法}以及\textbf{判断收敛性}

\subsection{雅可比迭代法、高斯-塞德尔迭代法}

题目给出$Ax=b$,转化成$x=Bx+g$,例如:

\[
\begin{cases}
5x_1 + 2x_2 + x_3 = -12 \\
-x_1 + 4x_2 + 2x_3 = 20 \\
2x_1 - 3x_2 + 10x_3 = 3
\end{cases}
\xrightarrow{}
\begin{cases}
x_1 = \frac{1}{5} \left(-2x_2 - x_3  -12\right) \\
x_2 = \frac{1}{4} \left(x_1 - 2x_3  +20\right) \\
x_3 = \frac{1}{10} \left(-2x_1 + 3x_2  +3\right)
\end{cases}
or
\begin{cases}
x_1 = -\frac{2}{5}x_2 - \frac{1}{5}x_3  -\frac{12}{5} \\
x_2 = \frac{1}{4}x_1 - \frac{1}{2}x_3  + 5 \\
x_3 = -\frac{1}{5}x_1 + \frac{3}{10}x_2  +\frac{3}{10}
\end{cases}
\]

左侧雅可比迭代法(轮换),右侧高斯-塞德尔迭代法(实时),取$x^{(0)}=(1,1,1)^T$计算

\[
\begin{cases}
x_1^{(k+1)} = -\frac{2}{5}x_2^{(k)} - \frac{1}{5}x_3^{(k)}  -\frac{12}{5} \\
x_2^{(k+1)} = \frac{1}{4}x_1^{(k)} - \frac{1}{2}x_3^{(k)}  + 5 \\
x_3^{(k+1)} = -\frac{1}{5}x_1^{(k)} + \frac{3}{10}x_2^{(k)}  +\frac{3}{10}
\end{cases}
\quad
\begin{cases}
x_1^{(k+1)} = -\frac{2}{5}x_2^{(k)} - \frac{1}{5}x_3^{(k)}  -\frac{12}{5} \\
x_2^{(k+1)} = \frac{1}{4}x_1^{\textcolor{red}{(k+1)}} - \frac{1}{2}x_3^{(k)}  + 5 \\
x_3^{(k+1)} = -\frac{1}{5}x_1^{\textcolor{red}{(k+1)}} + \frac{3}{10}x_2^{\textcolor{red}{(k+1)}}  +\frac{3}{10}
\end{cases}
\]

\subsection{判断收敛性}

四个方法按顺序判断

\noindent
1. 若$A$为\textbf{严格对角占优},则$J$和$GS$都收敛(充分)。

对角占优是指对角线上的元素绝对值大于同一行(列)其余元素绝对值之和,全部都大则是严格行(列)对角占优。行列一个是严格的就可以判断收敛。

\noindent
2. $B_j$存在某种范数\ref{fanshu}绝对值小于1,则$J$和$GS$都收敛(充分)。

$B_j$是对角线为0,其余元素除以对应行对角线元素再乘以-1

\noindent
3. 若$A$为正定矩阵,则$GS$收敛(充分),若$A$和$2D-A$都为正定矩阵,则$J$收敛(充要)。

正定矩阵:顺序主子式大于0,\quad $D$是$A$的对角线元素矩阵

\noindent
4. 谱半径 $\rho(B) < 1 \iff$ 则对应$J$和$GS$收敛(充要)。\quad $\rho(B)$是\textbf{特征值最大值},简便计算:

\[
|\lambda I - B_j| = 0 \implies
\begin{vmatrix}
\lambda a_{11} & a_{12} & \cdots & a_{1n} \\
a_{21} & \lambda a_{12} & \cdots & a_{2n} \\
\vdots & \vdots & \ddots & \vdots \\
a_{n1} & a_{n2} & \cdots & \lambda a_{nn}
\end{vmatrix}
= 0
\quad \quad
|\lambda I - B_S| = 0 \implies
\begin{vmatrix}
\lambda a_{11} & a_{12} & \cdots & a_{1n} \\
\lambda a_{21} & \lambda a_{12} & \cdots & a_{2n} \\
\vdots & \vdots & \ddots & \vdots \\
\lambda a_{n1} & \lambda a_{n2} & \cdots & \lambda a_{nn}
\end{vmatrix}
= 0
\]
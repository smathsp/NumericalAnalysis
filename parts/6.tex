\section{第6章 \quad 解线性方程的迭代法}

本章要求掌握\textbf{雅可比迭代法}、\textbf{高斯-塞德尔迭代法}以及判断收敛性

\subsection{雅可比迭代法、高斯-塞德尔迭代法}

题目给出$Ax=b$,转化成$x=Bx+g$,例如:

\[
\begin{cases}
5x_1 + 2x_2 + x_3 = -12 \\
-x_1 + 4x_2 + 2x_3 = 20 \\
2x_1 - 3x_2 + 10x_3 = 3
\end{cases}
\xrightarrow{}
\begin{cases}
x_1 = \frac{1}{5} \left(-2x_2 - x_3  -12\right) \\
x_2 = \frac{1}{4} \left(x_1 - 2x_3  +20\right) \\
x_3 = \frac{1}{10} \left(-2x_1 + 3x_2  +3\right)
\end{cases}
or
\begin{cases}
x_1 = -\frac{2}{5}x_2 - \frac{1}{5}x_3  -\frac{12}{5} \\
x_2 = \frac{1}{4}x_1 - \frac{1}{2}x_3  + 5 \\
x_3 = -\frac{1}{5}x_1 + \frac{3}{10}x_2  +\frac{3}{10}
\end{cases}
\]

左侧雅可比迭代法(轮换),右侧高斯-塞德尔迭代法(实时),取$x^{(0)}=(1,1,1)^T$计算

\[
\begin{cases}
x_1^{(k+1)} = -\frac{2}{5}x_2^{(k)} - \frac{1}{5}x_3^{(k)}  -\frac{12}{5} \\
x_2^{(k+1)} = \frac{1}{4}x_1^{(k)} - \frac{1}{2}x_3^{(k)}  + 5 \\
x_3^{(k+1)} = -\frac{1}{5}x_1^{(k)} + \frac{3}{10}x_2^{(k)}  +\frac{3}{10}
\end{cases}
\quad
\begin{cases}
x_1^{(k+1)} = -\frac{2}{5}x_2^{(k)} - \frac{1}{5}x_3^{(k)}  -\frac{12}{5} \\
x_2^{(k+1)} = \frac{1}{4}x_1^{\textcolor{red}{(k+1)}} - \frac{1}{2}x_3^{(k)}  + 5 \\
x_3^{(k+1)} = -\frac{1}{5}x_1^{\textcolor{red}{(k+1)}} + \frac{3}{10}x_2^{\textcolor{red}{(k+1)}}  +\frac{3}{10}
\end{cases}
\]

\subsection{判断收敛性}

四个方法按顺序判断

